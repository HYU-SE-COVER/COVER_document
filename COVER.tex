\documentclass[conference]{IEEEtran}
\IEEEoverridecommandlockouts
\usepackage{cite}
\usepackage{amsmath,amssymb,amsfonts}
\usepackage{algorithmic}
\usepackage{graphicx}
\usepackage{textcomp}
\usepackage{xcolor}
\usepackage{kotex}
\usepackage{tabularx}
\usepackage{supertabular,booktabs}
\usepackage{adjustbox}
\usepackage{enumitem}
\usepackage{romannum}
\usepackage{makecell}
\usepackage{multirow}
\usepackage{graphics}
\usepackage{subfigure}
\usepackage{float}
\def\BibTeX{{\rm B\kern-.05em{\sc i\kern-.025em b}\kern-.08em
    T\kern-.1667em\lower.7ex\hbox{E}\kern-.125emX}}

\pagenumbering{arabic}
\begin{document}

\title{COVER \\
- From Past to Present:\\
Harnessing Matter to Integrate Legacy Devices\\
into the Next-Gen Smart Home Ecosystem\\
}

\author{\IEEEauthorblockN{SONG WOOJUNG}
\IEEEauthorblockA{\textit{Dept. Information System} \\
\textit{Hanyang University}\\
Seoul, Republic of Korea \\
opusdeisong@gmail.com}
\and
\IEEEauthorblockN{KWON KITAE}
\IEEEauthorblockA{\textit{Dept. Information System } \\
\textit{Hanyang University}\\
Seoul, Republic of Korea \\
jrinonamu@gmail.com}
\and
\IEEEauthorblockN{KIM JINA}
\IEEEauthorblockA{\textit{Dept. Information System} \\
\textit{Hanyang University}\\
Seoul, Republic of Korea \\
drjina02@gmail.com}
\and
\IEEEauthorblockN{YEO DAKYUM}
\IEEEauthorblockA{\textit{Dept. Information System } \\
\textit{Hanyang University}\\
Seoul, Republic of Korea \\
yeodakyum@gmail.com}
}

\maketitle

\begin{abstract}
In recent times, the global smart home market has seen remarkable expansion, projected to cross \$164 billion by 2028. Driven by the desire for convenience, security, and energy savings, we're witnessing sophisticated device integrations in our homes. However, this growth isn't without its challenges. The array of smart home devices from different brands has fragmented the user experience, leading to a maze of applications for device control. Many third-party apps, despite their intentions, often don't mesh seamlessly with central control devices like smartphones, a challenge evident in platforms such as Apple Home and Google Home App . Additionally, a significant number of devices still lack IoT integration, which challenges expanding smart home market.[1]\\
Addressing these challenges, we introduce the LG Cover, an innovative solution designed to bridge gaps in the smart home ecosystem. We suggest specialized integration for all LG devices, resurrecting legacy devices through the Matter protocol, and ensuring consistent user experience across various brands. This system supports the integration of both Matter and ThinQ devices, and also supports older devices based on IR technology, revitalizing its utility within modern smart home setups.\\
To facilitate seamless integration, the LG Cover adopt a structured setup process involving the registration of the hub to the ThinQ app and the Matter API device. ThinQ supporting or IR supporting devices are connected to LG Cover with proprietary protocols, and LG Cover mediates the connected devices to Matter enabled controllers. Users are enabled to communicate with non-Matter-supporting devices through LG Cover using Matter API. With LG Cover's comprehensive approach, users can experience the convenience of a unified smart home ecosystem, irrespective of the device's age or brand.
\end{abstract}

\newpage

\large{Role Assignments}
\begin{table}[H]
\center
\begin{tabular}{m{1.7cm} m{1.4cm} m{4.0cm}}
\toprule
Roles & Name & Description\\
\midrule
User/Customer & SONG WOOJUNG &User/Customer plays the role of a various ages who needs support with an old LG devices that doesn’t support matter api. Despite the advancements in tech and the ubiquity of smart homes, not every device owned is compatible with the latest standards. This poses a challenge in achieving a cohesive experience, especially when most contemporary hubs primarily cater to the latest models and technologies. Therefore, the LG Cover plays an instrumental role in bridging this gap. While it is equipped to support the Matter API, its design philosophy is to be inclusive. The goal is to serve users like Alex, ensuring that their older, reliable LG devices don't become obsolete. This includes even those devices that don't support LG ThinQ.\\\\
\begin{tabular}[c]{@{}l@{}}Product\\Designer\end{tabular} & KWON KITAE & A Product Designer is responsible for understanding user needs and business objectives to conceptualize and design products that provide effective solutions; they engage in user research to gather insights, sketch initial design ideas, create wireframes and interactive prototypes, collaborate with cross-functional teams like product managers, engineers, and marketers to refine designs, ensure consistency in brand aesthetics and user experience across products, gather and act on feedback from user testing sessions. Also works with documentation that helps other roles communicate with each other more effectively.\\\\
\bottomrule
\end{tabular}
\end{table}
\newpage

\begin{table}[H]
\center
\begin{tabular}{m{1.7cm} m{1.4cm} m{3.8cm}}
\midrule
Software developer & KIM JINA & A software developer is responsible for collaborating with development manager, product owners, and product designer to understand software requirements, converting these into functional specifications, designing the software architecture based on these requirements, writing clean and maintainable code, ensuring software is scalable and robust, testing the software for bugs and inconsistencies, deploying software to production environments, maintaining and updating software as necessary, and documenting development processes and software functionalities to ensure continuity and ease of future modifications. Also works with connection between the matter API and ThinQ devices to maintain connectivity.\\\\
Development manager & YEO DAKYUM & A Development Manager oversees the software development process, coordinating with cross-functional teams to ensure projects are completed on time and within budget, establishing and implementing development methodologies, setting objectives and deliverables for the development team, mentoring and providing guidance to developers, ensuring code quality and adherence to standards. Also works with the actual hardware chip that will be used in Cover.\\
\bottomrule
\end{tabular}
\end{table}

\section{\large{Introduction}}
\subsection{\large{Motivation}}
\begin{itemize}
\item Growth of Smart Home Systems\\
The smart home market has experienced remarkable growth over the last few years. According to Statista, the global smart home market is projected to surpass \$141 billion by 2023, up from \$76.6 billion in 2018.[2] This exponential growth is driven by consumers' increasing desire for convenience, security, and energy efficiency. As homes integrate more devices, the need for a cohesive control system becomes critical.\\
\item The Challenge of Diverse Devices from Different Companies \\
As smart home devices proliferate, a significant challenge arises due to the sheer variety of brands and products available. Different companies offer their distinct range of smart devices, each with its dedicated application for control. This leads to an overflow of applications on users' smartphones or tablets, complicating what should be a simplified process.[3][4] The fragmentation of control interfaces creates inefficiency and reduces the user-friendly nature of the smart home system. There's a clear need for a centralized platform, like Matter, that can consolidate control, ensuring that users can manage their entire smart home ecosystem without juggling multiple applications. \\
Additionally, there is a potential risk of user's personal information being exposed when remotely controlling products of diverse companies through IoT. Currently, when a user issues a command to turn on the lights from a mobile phone, it often goes through a cloud-based connection between the phone and the hub, and then the operation is executed via an API.[5] With Matter, all operations can be performed locally without passing through external servers, ensuring that personal information is not exposed.\\
\item Bridging the IoT Gap with LG Cover\\
While many modern devices are equipped with IoT capabilities, a substantial portion remains without this integration. The disparity between IoT-enabled devices and those lacking this feature can result in gaps within the smart home ecosystem. Matter, a unifying protocol for smart home devices, can introduce IoT capabilities to previously incompatible devices. LG Cover, supporting Matter, acts as a bridge, allowing even non-IoT devices to join the integrated smart home network. This inclusivity ensures that no device is left out, offering users a comprehensive smart home experience.\\
\end{itemize}

\subsection{\large{Problem Statement}}
\begin{itemize}
\item Gap in Dedicated Support for LG Devices\\
The landscape of smart home devices is increasingly complex and fragmented, with a myriad of brands and protocols available in the market. Within this expansive ecosystem, while there are many generic hubs available, there is a conspicuous absence of a dedicated hub designed specifically for LG's vast array of devices. Owners of LG products, both modern and legacy, find themselves in need of a hub that is acutely attuned to the nuances of LG's electronic architecture. Generic solutions often fall short in leveraging the full potential of LG devices, leaving users with a less-than-optimal experience.\\
\item Neglect of Legacy IR Devices and Old ThinQ Products\\
The tech industry's rapid advancement has led to a situation where many older, yet perfectly functional devices, especially those based on infrared (IR) technology and earlier versions of LG's ThinQ products, are left out of the modern smart home narrative. Most current hubs are squarely focused on the latest communication protocols, sidelining these older devices. This presents a challenge for many households that still rely heavily on these devices and are looking for ways to integrate them into a unified smart home ecosystem.\\
\item Diverse Brand Environment Leading to Fragmented User Experiences\\
The average household's smart device portfolio often comprises products from a variety of brands. Current IoT hubs, while claiming broad compatibility, tend to offer a disjointed user experience due to inconsistencies in how different brands' devices are integrated and managed. Users are thus left longing for a singular platform that can offer a seamless and harmonious experience across all their devices.\\
\end{itemize}

\subsection{\large{Solution}}
\begin{itemize}
\item Tailored Integration for LG's Spectrum of Devices\\
The LG Cover stands out with its commitment to providing a seamless experience specifically for LG device owners. Whether it's the latest smart TV or a legacy LG appliance, the LG Cover ensures perfect harmony and optimized performance, something that can only be achieved with a deep understanding of LG's technological DNA.\\
\item Inclusivity for Legacy Devices\\
One of LG Cover's hallmark features is its pioneering capability to revitalize older devices using the Matter protocol. Whether it's an IR-based remote-controlled device or an older iteration of the ThinQ product line, the LG Cover, through its integration with Matter, ensures that these devices are not only recognized but also seamlessly incorporated into the modern smart home ecosystem. By leveraging the universal nature of Matter, LG Cover effectively bridges the gap between past and present, preserving the utility and value of legacy devices in today's interconnected world.\\
\item Consistent Experience Across a Multitude of Brands\\
Beyond its specialization in integrating LG products, the LG Cover showcases its prowess in providing a consistent and intuitive interface, even when dealing with devices from multiple manufacturers. This focus on providing a unified user experience ensures that managing a diverse smart home setup is no longer a daunting task, but a delightful experience utilizing Matter, it would enable us to incorporate our device to many readily available, existing home management application, such as Apple HomeKit or Google Home App, which is much well suited for corresponding mobile OS.\\
\end{itemize}

\subsection{\large{Related Software}}
\begin{itemize}
\item Google Nest Hub 2 Gen.\\
Nest Hub is the device to use to manage your smart home since it is compatible with so many different things.\\
We can watch a range of movies, music, and TV shows using Nest Hub, and we can use voice commands or a single tap to operate connected smart devices. A sleep sensor function can also promote restful sleep.\\
Many Nest Hub functionalities are accessible through Quick Gestures. Quick Gestures uses Motion Sense instead of a camera to identify when our hand moves.\\
\item Samsung SmartThings Station\\
Samsung SmartThings Station combines a smart home hub and wireless phone charger while supporting up to three separate programmed automation routines.\\
It works with a broad selection of Matter and other Samsung smart home products, and its accompanying app makes it simple to set up routines you can activate from the Station with a tap.\\
It is incapable of playing music or supporting voice control because it lacks a speaker and microphone. Instead, this hub serves as a wireless charging station for devices that are Qi compliant.\\
Both the Android and iOS platforms currently support Samsung SmartThings Station.\\
\item SwitchBot Hub 2\\
SwitchBot Hub 2 supports old infrared appliances, we can link additional SwitchBot goods through Bluetooth and Wi-Fi to create a smart home environment.\\
We can integrate Hub 2 into our smart home ecosystem to deploy Matter over Wi-Fi in the future for current Bluetooth-only products like SwitchBot Curtain, SwitchBot Lock, and SwitchBot Blind Tilt, eliminating the need to purchase additional hardware to make them Matter compatible.\\
\item Ikea Dirigera Hub\\
DIRIGERA hub is the central control hub for a smart home in the IKEA Home smart app.\\
It enables us to connect and control a variety of smart gadgets, including speakers, blinds, and lighting. These devices allow for remote control, the creation of personalized settings for various situations and emotions, and the scheduling of automated chores.\\
We can operate your smart home using the app, remotes, voice commands, or motion thanks to the hub's flawless integration into our house's design. It is made to be user-friendly for everyone, including visitors, of all ages.\\
IKEA offers a growing selection of smart items to expand our setup over time and updates the app often to improve the smart home experience.\\
\item Bosch Smart Home Controller 2nd Gen.\\
Bosch Smart Home Controller II is a crucial part of the Bosch Smart Home ecosystem, serving as both hardware and software hub. It connects to various smart devices like thermostats and cameras and can integrate with third-party products. Users control it through a user-friendly app, web interface, or voice commands via popular voice assistants. It communicates using Wi-Fi, Zigbee, and Z-Wave.\\
Security and privacy are a priority with strong encryption. Regular updates enhance performance and security. It's energy-efficient and scales easily, making it the central unit for smart homes, providing control, security, and efficiency.\\
Smart Home Controller II ensures the security of our data, even when accessed remotely while we’re on the go. We have the flexibility to either control our system exclusively within our home network or remotely to monitor its status.\\
\item Homebridge\\
Homebridge is a lightweight NodeJS server that emulates the iOS HomeKit API. It allows users to integrate various smart home devices that do not natively support Apple’s HomeKit into the HomeKit ecosystem. Homebridge facilitates a broad array of plugins from various developers, which enables the connection and control of non-HomeKit devices via your iOS devices.\\
Being open-source software, Homebridge allows for transparency, community input, and modification. Users can delve into the code, understand how it works, and even contribute to its development. This openness fosters innovation and enables continuous improvement of the platform.\\
Its open-source nature, extensive plugin support, and active community make it a unique and invaluable tool for smart home enthusiasts, especially those invested in the Apple ecosystem.\\
\end{itemize}

\section{\large{Requirements}}
\subsection{\large{Installing Hub}}
\begin{enumerate}[label=\arabic*.]
\item {\large{Register LG Cover to ThinQ}}\\
To use non-Matter-supportive devices on home assistants as if they were Matter devices, LG Cover is to be registered to ThinQ app. The information regarding the device control function will be given to LG Cover to generate Matter-supportive QR code.\\
Registration procedure is as follows:\\
\begin{enumerate}[label=\alph*.]
\item In the ThinQ app, press ‘+’ to add new devices. A camera screen that reads QR codes appears.\\
\item Above the square frame for QR codes, click ‘matter’ button to add the LG Cover hub. \\
\item Read the QR code on LG Cover’s screen to register the hub to ThinQ app. \\
\item LG Cover is set to ThinQ app.\\
\end{enumerate}

\item {\large{Register LG Cover to Matter API device}}\\
LG Cover needs to be connected to user’s Matter API device to provide information from ThinQ app to home assistants such as Google Home, Apple Home, Alexa, and more. Matter API devices will recognize LG Cover as a Matter device and send or receive signals upon Matter protocols.\\
Registration procedure is as follows:\\
\begin{enumerate}[label=\alph*.]
\item In the home assistant application of the user’s device, execute menu for adding new devices.\\
\item Among the various ways and brands of devices, select ‘Matter’ or ‘Add Matter device’. A camera screen that reads QR codes appears. \\
\item Read the QR code on LG Cover’s screen and click “Add to Home Assistant”. \\
\item LG Cover is set to the user’s Matter API devices.\\
\end{enumerate}
\end{enumerate}

\subsection{\large{Register device to LG Cover}}
Devices registered to ThinQ app can be operated from user’s home assistant just like Matter devices via LG Cover. ThinQ and LG Cover is connected, and LG Cover, the bridge hub, mediates ThinQ and Matter API home assistants. Therefore, as users register their devices to Register devices to LG Cover using one of the following three methods according to the supported module for the device.\\
\begin{enumerate}[label=\arabic*.]
\item {\large{Matter device}}\\
All matter devices use setup code, which is provided on the device, in the packaging, or in an app. Register the Matter device to ThinQ to control device though LG Cover. Matter devices can be registered to the Matter API device through ThinQ or directly to home assistants.\\
Registration procedure is as follows:\\
\begin{enumerate}[label=\alph*.]
\item In the ThinQ app, press ‘+’ to add new devices. A camera screen that reads QR codes appears.\\
\item Above the square frame for QR codes, click ‘matter’ button to add device.\\
\item Read the QR code on the device to register the device to ThinQ app. \\
\item Click “Connect through NFC” or “Connect through Pin” to connect the device by tapping on the NFC tag or manually entering the setup digits.\\
\item Device is registered to ThinQ.\\
\end{enumerate}
Registration procedure to home assistant is as follows:\\
\begin{enumerate}[label=\alph*.]
\item Read the QR code on the device.\\
\item Page will be redirected to registration page for home assistant app.\\
\item Click “Add to Home Assistant”. \\
\item Device is registered to Matter API home assistant.\\
\end{enumerate}

\item {\large{ThinQ device}}\\
LG appliances that support ThinQ can be registered easily on the app.  Devices connected to ThinQ are automatically added to Matter devices via LG Cover acting as a bridge. Register the device to ThinQ to manage the device from home assistants through LG Cover.\\
Registration procedure is as follows:\\
\begin{enumerate}[label=\alph*.]
\item In the ThinQ app, press ‘+’ to add new devices. A camera screen that reads QR codes appears.\\
\item Read the QR code of the device or find the device from the list of products provided.\\
\item Device is registered to ThinQ app.\\
\end{enumerate}

\item {\large{Legacy Infrared(IR) device}}\\
Infrared supporting devices could or could not be LG’s products. If the device is not in LG’s database, IR signals should be recorded with IR receiver. If the device is LG’s product, IR signals would be listed in LG’s database. In this context, LG Cover must possess the capability to access the database of infrared signals associated with a particular device.\\
Devices that supports only infrared remote control can be registered to ThinQ and be managed by LG Cover.\\
Registration procedure is as follows:\\
\begin{enumerate}[label=\alph*.]
\item In the ThinQ app, click "Add device by IR". \\
\item ThinQ requests for IR signals.\\
\item User sends signals to LG Cover through remote controllers or any kind of IR controllers.\\
\item The name of the device will be set by smart matching or manually by user.\\
\item The device is registered to ThinQ.\\
\end{enumerate}
\end{enumerate}

\subsection{\large{Integration}}
LG cover is able to access ThinQ API via RESTful API. As new devices that do not support Matter protocol are registered to ThinQ app, LG Cover pulls users’s devices from ThinQ. The information can hold the name of the device and list of signals for device operation to be managed in Cover.\\
LG Cover receives information about the newly registered device. LG Cover is already linked to Matter API devices, or home assistants. Therefore, LG Cover mediates the newly registered device to home assistants and lets Matter API devices to recognize the device as a Matter-supporting device and communicate with the device using matter signals.\\
The user's actions to trigger these operations and the corresponding signal flow are as follows:\\
\begin{enumerate}[label=\arabic*.]
\item {\large{Matter device}}\\
Devices that support Matter can be controlled directly from home assistants using Matter API.\\

\item {\large{ThinQ device}}\\
Communication from user to device can be processed as following:\\
\begin{enumerate}[label=\alph*.]
\item User sends a Matter signal through home assistant.\\
\item LG Cover receives the Matter signal.\\
\item LG Cover delivers the command to ThinQ API by network.\\
\item ThinQ app receives the command from LG Cover and executes the command.\\
\end{enumerate}
Communication from device to user can be processed as following:\\
\begin{enumerate}[label=\alph*.]
\item Device sends a signal to ThinQ through network.\\
\item ThinQ receives the signal from device and LG Cover acquires the data from ThinQ API.\\
\item LG Cover translates the signal and delivers the data to home assistant by Matter signal.\\
\item Matter API device, home assistant, receives the data and conveys information.\\
\end{enumerate}

\item {\large{IR device}}\\
Communication from user to device can be processed as following:\\
\begin{enumerate}[label=\alph*.]
\item User sends a Matter signal through home assistant.\\
\item LG Cover receives signal via Matter.\\
\item LG Cover indicates the Matter signal and transmits the corresponding IR signal for IR device to receive.\\
\end{enumerate}
\end{enumerate}

\subsection{\large{Control device using LG Cover}}
\begin{enumerate}[label=\arabic*.]

\item {\large{Matter device}}\\
Matter devices provide simple and complicated control options available by Cover and home assistants.\\
Cluster is a group of functions, such as On/Off Cluster or Level Control Cluster. In general, user should be able to use a variety of clusters supported by home assistant, along with manufacturer-specified functions. Among the clusters supported by home assistant, some of the most common ones include On/Off, Open/Close, Color Control, Level Control, Temperature Measurement, and more.\\
Node encapsulates the full product and controls all endpoints of the particular device. Each endpoint includes clusters, and each cluster holds attributes and events.\\
Consider a device that belongs to On/Off Light Switch Device Type, and holds two endpoints. Endpoint 1 includes the Server On/Off Cluster, and Endpoint 2 includes the Client On/Off Cluster.\\
The flow of an On/Off control is as follows:\\
\begin{enumerate}[label=\alph*.]
\item User clicks the On/Off button from home assistant.\\
\item The Client Cluster controls the Server Cluster.\\
\item Server Cluster triggers the command to turn on or turn off the switch.\\
\begin{enumerate}[label=\roman*.]
\item If the OnOff attribute of the Node is “On”, the switch would be turned off.\\
\item If the OnOff attribute of the Node is “Off”, the switch would be turned on.\\
\end{enumerate}
\end{enumerate}

\item {\large{ThinQ device}}\\
Complicated control functions of appliances can also be added to Matter API device in forms of shortcuts. Cover enables user selected commands be registered as shortcuts in the device, and user is able to execute the command by just one click.\\
Simple Control:\\
\begin{enumerate}[label=\alph*.]
\item Turn On / Turn Off\\
User can turn on and turn off the matter device from home assistants by clicking “On/Off” button from the home assistant app or from user defined shortcuts.\\
\end{enumerate}
Complicated Control:\\
\begin{enumerate}[label=\alph*.]
\item Refrigerator Temperature\\
User can check the temperature of each section through home assistants.\\
User can set the temperature to personal preference through home assistants.\\
\item Washing Machine\\
Washing mode, water temperature, number of rinse, and other control functions can be set with home assistants.\\
\end{enumerate}

\item {\large{IR device}}\\
\begin{enumerate}[label=\alph*.]

\item LG product\\
LG holds information of all LG products. Therefore, all commands of IR devices are available even if the device does not support ThinQ. When LG IR device is registered to ThinQ through LG Cover, ThinQ can manage several commands of the device with LG Cover for Matter controls.\\
Simple Control:\\
\begin{enumerate}[label=\roman*.]
\item Turn On / Turn Off\\
User can turn on and turn off the matter device from home assistants by clicking “On/Off” button from the home assistant app or from user defined shortcuts.\\
\end{enumerate}
Complicated Control:\\
\begin{enumerate}[label=\roman*.]
\item Air conditioner
\begin{enumerate}
\item Wind direction\\
Direction of the be wind, from top to bottom, left to right, can set through home assistants.\\
\item Wind temperature\\
Increase and decrease in wind temperature can be set through home assistants.\\
\item Operation Mode\\
Operation mode such as Air-conditioning, dehumidification, and heating can be set through home assistants.\\
\end{enumerate}
\item TV
\begin{enumerate}
\item Remote Controller\\
Sound, channels, and other control functions can be set with home assistants.\\
\end{enumerate}
\end{enumerate}

\item Non-LG product\\
Information of IR devices of other manufacturers is not available by LG. Therefore, when the device is registered to ThinQ through LG Cover, information to be sent is very limited. Only simple control functions can be registered to Cover.\\
\begin{enumerate}[label=\roman*.]
\item Turn On / Turn Off\\
User can turn on and turn off the matter device from home assistants by clicking “On/Off” button from the home assistant app or from user defined shortcuts.\\
\end{enumerate}

\end{enumerate}

\end{enumerate}

\subsection{\large{Default display of LG Cover}}
When the LG Cover is not in progress of connecting devices, LG Cover displays date and time, temperature, and humidity on the screen.\\

\begin{thebibliography}{00}
\bibitem{b1} Smart Home Market Revenue Trends and growth drivers - 2023. MarketsandMarkets. https://www.marketsandmarkets.com/Market-Reports/smart-homes-and-assisted-living-advanced-technologie-and-global-market-121.html?gclid=EAIaIQobChMImsWklrrwgQMV62sPAh2Blg4VEAAY\\ASAAEgLVQfD\_BwE \\
\bibitem{b2} 2019 Digital America State of the U.S. Consumer Electronics Industry. https://cdn.cta.tech/cta/media/media/resources/i3/pdfs/digital-america-2019.pdf \\
\bibitem{b3} Ding, J., Nemati, M., Ranaweera, C., \& Choi, J. (2020). IoT Connectivity Technologies and Applications: A Survey. IEEE Access, 8, 67646-67673.\\
\bibitem{b4} Shah, S.K., \& Mahmood, W. (2020). Smart Home Automation Using IOT and its Low Cost Implementation. International Journal of Engineering and Manufacturing, 10, 28-36.\\
\bibitem{b5} Iqbal, W., Abbas, H., Daneshmand, M., Rauf, B., \& Bangash, Y.A. (2020). An In-Depth Analysis of IoT Security Requirements, Challenges, and Their Countermeasures via Software-Defined Security. IEEE Internet of Things Journal, 7, 10250-10276.
\end{thebibliography}

\end{document}
